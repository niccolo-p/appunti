\documentclass[italian,a4paper]{article}
\usepackage{babel,a4,amsmath,amssymb,amsthm}
\newcommand{\R}{\mathbb{R}}
\newcommand{\N}{\mathbb{N}}
\newcommand{\Z}{\mathbb{Z}}
\renewcommand{\epsilon}{\varepsilon}

\newtheorem{theorem}{Teorema}
\newtheorem{proposition}[theorem]{Proposizione}
\newtheorem{example}[theorem]{Esempio}
\newtheorem{corollary}[theorem]{Corollario}
\newtheorem{lemma}[theorem]{Lemma}
\newtheorem{definition}[theorem]{Definizione}
\newtheorem{exercise}[theorem]{Esercizio}

\title{La successione $\sin n$}
%%title{La successione $\sin n$}
\author{Emanuele Paolini}
\date{11 novembre 2004}
\begin{document}
\maketitle
In questa nota mostreremo che l'insieme dei punti della successione $a_n=\sin n$ \`e denso nell'intervallo $[-1,1]$.

\begin{proposition}
Siano $\alpha,\beta$ numeri reali non nulli tali che $\alpha/\beta$
sia irrazionale. Allora per ogni $N\in\N$ esistono $p,q\in\Z$ con
$|p|\le N$ tali che
\[
  0 \neq |p\alpha + q\beta| \le 1/N.
\]  
\end{proposition}
\begin{proof}
Dato $p\in \Z$ consideriamo il numero intero
\begin{equation}\label{defq}
   q := -\left\lfloor\frac{p\alpha}\beta\right\rfloor.
\end{equation}
Chiaramente si ha $q\in[-p\alpha/\beta,1-p\alpha/\beta)$ da cui si
nota che con questa scelta di $q$ vale
\[
p\alpha + q\beta \in [0,1).
\]
Consideriamo ora gli $N+1$ numeri
$p=0,1,\ldots,N$. Scegliamo i corrispondenti valori di $q$ dati
da~\eqref{defq}.
Otteniamo dunque $N+1$ numeri $p\alpha + q\beta$ tutti compresi
nell'intervallo $[0,1)$. 
Dunque di questi $N+1$ numeri, ce ne devono essere sicuramente due che
distano tra loro meno di $1/N$. Dunque esistono due interi distinti
$p_1$ e $p_2$ e i corrispondenti $q_1$ e $q_2$ tali che
\[
  \lvert p_1\alpha+q_1\beta - (p_2\alpha + q_2\beta)\rvert \le 1/N
\]
da cui posto $p=p_1-p_2$, $q=q_1-q_2$ si trova che $|p|\le N$ e 
\[
  \lvert p\alpha + q\beta\rvert \le 1/N.
\]

Ci resta da dimostrare che $p\alpha + q\beta \neq 0$. Siccome $p_1$ e
$p_2$ erano distinti per costruzione, sappiamo che $p\neq 0$. Se per
assurdo fosse $p\alpha+q\beta =0 $ avremmo $\alpha/\beta=-q/p$ che
contraddice l'ipotesi secondo cui $\alpha/\beta$ non \`e razionale.
\end{proof}
\begin{corollary}
Siano $\alpha,\beta$ numeri reali non nulli tali che $\alpha/\beta$
sia irrazionale. Allora, dato qualunque $x\in\R$ e qualunque
$\epsilon>0$ esistono $p,q\in \Z$, $p\ge 0$ tali che 
\[
  0\neq \lvert p\alpha+q\beta - x\rvert < \epsilon.
\]
\end{corollary}
\begin{proof}
Sia $N>2/\epsilon$ e siano $p_0,q_0\in \Z$, 
dati dalla proposizione precedente, tali che
\[
  0\neq |p_0 \alpha + q_0\beta| \le 1/N < \epsilon/2.
\]
Supponiamo anche $p_0\ge 0$ (in caso contrario basta cambiare segno ad
entrambi $p_0$ e $q_0$).
Posto $\delta:= |p_0 \alpha + q_0\beta|$ abbiamo dunque
$\delta\in(0,\epsilon/2)$ e scelto $k=\lceil x/\delta -1\rceil$ troviamo
che $k\delta\in(x-2\delta,x-\delta)$. Dunque con la scelta $p=kp_0$ e
$q=kq_0$ troviamo $p,q\in Z$, $p\ge 0$ e 
\[
  p\alpha + q\beta = k\delta \in (x-2\delta,x-\delta)\subset (x-\epsilon,x)
\]
da cui segue immediatamente la tesi.
\end{proof}

Con questo lemma a disposizione siamo pronti a dimostrare che
l'insieme dei punti limite della successione $a_n=\sin n$ \`e
l'intervallo $[-1,1]$. 
Dato $\alpha\in[-1,1]$ 
consideriamo infatti un numero $x$ tale che
$\sin x=\alpha$. Preso un qualunque $k\in N$, 
applicando il corollario precedente con $\alpha=1$ e
$\beta=2\pi$ possiamo trovare due numeri interi $p_k,q_k\in\Z$ con $p_k>0$
tali che 
\[
  0\neq |p_k - (x-2\pi q_k)| < 1/k.
\]
Dunque, sapendo che $|\sin x-\sin y|\le|x-y|$, si ha  
\[
  |\sin p_k-\alpha| = |\sin p_k - \sin(x-2 q_k\pi)| 
\le |p_k-(x-2\pi q_k)|\le 1/k.
\]
Dunque $p_k$ \`e una successione di numeri interi tali che $\sin
p_k\to \alpha$. 
Siccome $|p_k - (x-2\pi q_k)|\neq 0$ possiamo facilmente concludere
che $p_k$ assume infiniti valori distinti e quindi
$p_k\to+\infty$. Dunque esiste una sottosuccessione strettamente
crescente $p_{k_n}$. La successione corrispondente $a_{p_{k_n}} =
\sin(p_{k_n})$ converge dunque ad $\alpha$.

\end{document}
